\documentclass[a4paper,12pt]{article}
\usepackage[T1]{fontenc}
\usepackage [isolatin]{inputenc} % fontes avec caracteres accentues
\usepackage{graphicx} % inclusion de figures
\usepackage{listings}

\begin{document}

\title{\emph{Gaspard Documentation}}
\author{Nicolas de S\'er\'eville}

\maketitle 
\pagebreak
\tableofcontents % la table des matieres
\pagebreak


\section{Introduction}
The Gaspard project is developed within the NPTool framework. For the 
moment only the tracker of charged particles is currently under study.
Coupling of the tracker with a gamma-ray calorimeter such as AGATA or
PARIS will be considered in the future. 

NPTool is a modular package allowing to perform Geant4 simulations and to 
analyse the results of the simulations. It is strongly encouraged to read 
the general NPTool documentation that you can find in this directory.


\section{NPSimulation}
\subsection{Specificity of Gaspard}
The Gaspard tracker detector, even if it is made of several types of detectors, 
is considered as {\it one} detector from the NPTool point of view.

\subsection{Running the simulation}
\subsection{Results of the simulation}
The results of the simulation are in the ROOT format and the output file 
is stored in \$NPTool/Output/Simulation. The output ROOT file contains
three classes:
\begin{itemize}
   \item {\it TInitialConditions:}
   \item {\it TInteractionCoordinates:}
   \item {\it TGaspardTrackerData:}
\end{itemize}

\subsection{Adding a new detector shape to Gaspard}
A special class (GaspardTrackerDummyShape) has been created to show how
to define a new module for the Gaspard Tracker. This class describes a
simple 5x5 cm2 square telescope made of three layers of silicon which  
has been used for some preliminary studies of the tracker. So, when 
considering adding a new module to the Gaspard Tracker, please do not use
this class but create you own instead. However, for the explanations the 
GaspardTrackerDummyShape case will bed considered.

When adding a new detector you need to follow several steps:

\subsubsection{Definition of an index for the detector}
Since the results of the simulation are stored in an unique data class 
(GaspardTrackerData) dealing with different module types, it is 
necessary to give an absolute number to each module. This is managed in
the GaspardTrackerModule class where there is a map (m\_index) which
associates a name (the module type identifier) with an integer (the value 
of the index).

To add a new detector it is just needed to add in the {\it InitializeIndex()}
method a line such as:

\begin{verbatim}
   m_index["DummyShape"] = 1000;
\end{verbatim}

\subsubsection{Definition of the geometry and its readout}
This is done in two files GaspardTrackerDummyShape.{h,cxx}. The geometry 
is defined in the {\it VolumeMaker()} method, the placement of the 
module is defined in the {\it ConstructDetector()} method, the readout
of the geometry is done in the {\it ReadSensitive()} method and the 
scorers are declared in the {\it InitializeScorers()} method.

\subsubsection{Definition of the scorers}


\section{NPAnalysis}
To be documented...

\end{document}

